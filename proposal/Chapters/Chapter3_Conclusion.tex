\chapter{Conclusion} \label{conclusion}
Overall, this background study has covered the recent increase of data worldwide, the difference between parallel and distributed computing, existing and proposed first-class constructs that are essential for distributed computing in C++, common issues with memory in distributed applications, a birds-eye view of Application Binary Interfaces, and many projects with similar goals to this one. 


In the related work section, there was astudy of C++ frameworks implementing distributed data structures and algorithms along with efficient distributed memory systems. While this document is nowhere near a complete study of the field, it provides sufficient entry level reading on this subject. There are numerous frameworks that could not be included for the sake of space, including, but not limited to, SPLIT-C, POOMA, CILK, NESL, and Titanium. Focus was placed on projects that are in continued development instead of those projects that have longsince been lacking a release, excluding many other frameworks. Additionally, many distributed memory applications were excluded for sake of brevity. redis, for example, is a more full version of memcached, with many of the same features and improved data structures. There are also many other applications for storing key-value pairs including Hadoop, SQL variations, MongoDB, etc., but many of these applications gained inspiration from memcached, explaining the space it takes up in this study. 

\paragraph{Future work}
This background document will certainly not be the end of research for this thesis project, as new advances in distributed computing are published every day, but it will serve as a sufficient launching pad to get started. Future efforts will certainly consider any novel methods used in the frameworks/distributed memory applications mentioned above. A further in-depth study of the specific methods that will be used in a given Application Binary Interface is needed before any work in that area is done. Additionally, novel methods that modify the Linux kernel in order to yield better performance from big memory applications like \cite{virtual_memory_tlb} could be useful, but it remains to be seen whether they are within the scope of this project - making significant changes to the Linux kernel, however small they are, may prove to be too much for a thesis project by a single student and his advisor. Project managers often categorize knowledge in a matrix: known knowns, unknown knowns, known unknowns, and unknown unknowns. The subjects I listed are all known unknowns, but there are also many unknown unknowns, or pieces of information which we do not know that we do not know, which may prove relevant enough to be added to this background document throughout the course of the project. 
